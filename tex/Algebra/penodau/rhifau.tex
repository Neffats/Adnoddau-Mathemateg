\chapter{Rhifau}

\section{Y rhifau naturiol ac Anwythiad}

Mae'r rhifau naturiol yn elfennau o'r set:

\begin{equation}
  \mathbb{N} := \{1, 2, 3, \ldots\},
\end{equation}

Yn y llyfr hon, nid ydynt yn ystyried $0$ fel rhan o'r set yma. Er
mwyn cyfeirio at y set o rifau naturiol gan gynnwys $0$, rydym yn
ysrifennu:

\begin{equation}
  \mathbb{N}_0 = \{0, 1, 2, \ldots \}.
\end{equation}


\paragraph{Gwirebau Peano}
\begin{enumerate}
\item Mae $1$ yn rif naturiol.
\item Os ydy $n$ yn rif naturiol, yna bodolir yn union un olynwr,
  dynodir gan $n+1$, sydd hefyd yn rif naturiol.
\item Nid ydy $1$ yn olynnu unrhyw rif naturiol.
\item Mae gan rifau naturiol gwahanol olynwyr gwahanol.
\item (\textit{Egwyddor Anwythiad}) Gadwch i $P(n)$ fod yn briodwedd o
  rif naturiol $n$. Os ydy $P(1)$ yn wir, yna os ydy'r tybiaeth fod
  $P(n)$ yn wir am ryw rif naturiol $n$ yn golygu fod $P(n+1)$ hefyd
  yn wir, mae $P(n)$ yn wir am bob rhif naturiol $n$.
\end{enumerate}